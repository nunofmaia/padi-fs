 
%
%  $Description: Author guidelines and sample document in LaTeX 2.09$ 
%
%  $Author: ienne $
%  $Date: 1995/09/15 15:20:59 $
%  $Revision: 1.4 $
%

\documentclass[times, 10pt,twocolumn]{article} 
\usepackage{padi-fs}
\usepackage{times}

%\documentstyle[times,art10,twocolumn,latex8]{article}

%------------------------------------------------------------------------- 
% take the % away on next line to produce the final camera-ready version 
\pagestyle{empty}

%------------------------------------------------------------------------- 
\begin{document}

\title{PADI-FS}

\maketitle
\thispagestyle{empty}

\begin{abstract}
   PADI-FS is a simple distributed file system that allows users to manage
   sets of files. Users that want to have their files replicated on several
   machines and to be always available can use PADI-FS to fulfill their
   needs and wishes. The system provides to the user some simples operations
   for them to manipulate their files as if they were on their local machine.\\

   This paper describes the architecture of the system being developed, the
   solutions to the problems we can face when building this kind of system
   and the reasons that led us to those same solutions, comparing them to
   other possible solutions we thought were not appropriate for this system.
\end{abstract}



%------------------------------------------------------------------------- 
\section{Introduction}

Computers use a file system to store, retrieve and update files. The file
system provides an interface to manage the organization and the access of
the data stored in it.

A distributed file system extends the regular file system allowing the
access to files from multiple machines. This kind of system makes possible
for a large number of users on their own machines to access the same files.

There are lots of implementations of distributed systems, such as AFS 
(Andrew File System), Google File System and HDFS (Hadoop Distributed File 
System). With the implementation of PADI-FS, we aim to do something very
similar but a lot simpler.

%------------------------------------------------------------------------- 
\section{Architecture}

The PADI-FS architecture is composed by three main components. First, the
metadata server keeps the file's metadata and is replicated by three
replicas. Second, the data server is responsible for storing the contents
of the files. Exists more than one data server, allowing the contents of
a given file to be replicated. And third, the client is responsible for
making the requests.\\

The following figure is meant to illustrate the architecture of the system.\\

\begin{figure}[h]
   \caption{System architecture.}
\end{figure}

These three components will be described in the sub-sections below to give
a deep insight on the way each component works and how it's implementation
solves the different problems that can occur.

There is also a fourth component of the system, denominated Puppet Master,
that is a centralized controller that can issue commands to all other
components. Because this component only exists to simplify testing and
debugging of the system, it's not described in this paper.

%------------------------------------------------------------------------- 
\subsection{Metadata Server}

Metadata servers maintain metadata information about the files stored
in the data servers. This metadata holds relevant information about
the files such as the file name, the number of servers used to store
the contents of the file, the minimum number of data server that should
be contacted during the read and write operations and a list of the
data servers that have the file stored. The metadata server also stores
information about the number of files each data server stores, which
servers are available or not and which files are open in clients.
The reason why we do this will be explained below on document.

As it was previously mentioned, there are 3 replicas of the metadata
server. The purpose of these replicas is to have the content of the
server replicated so that when a replica fails, there's another one
that can reply to the requests made. This replication is made in a
passive manner, which means that just one of the replicas will be
responsible for all the operations. The primary replica has to
synchronize its state with the secondary ones to make sure that all
the replicas store the same information at all time. Passive replication
is more useful in this system than active replication in a way that we
can tolerate two failures and have always a server to be able to reply to
any request made by the client.

\subsubsection{Handling Files}

The primary server can reply to several requests from the client. This
request are create, open, close and delete operations.
When a client send a request to create a file, the server saves a new
registry for that file with the information sent with the request. The
primary replica must inform the necessary data servers that they need to
create a local copy of the file and wait for them to reply back. Then
it increases the number of files for each server and sends the new 
entry to the other two replicas so they can be up to date. It's important 
to mention that when a data server is down, the metadata server will try 
to send the request to another server. If there are no more servers to 
send the request to, the metadata server will wait for the data server 
to come up to send it the information it needs to create the file. A 
secondary replica can also be down when the primary one sends its new 
information. When this happens, the primary server will not wait for 
the reply of the other server. Just after this exchange of requests and
replies, the metadata server tells the client it's all good and the file 
was successfully created.

When a client asks to delete a file, the metadata server decreases the
number of files for each server that holds the file and removes the 
corresponding entry from the table where all the information is stored. 
The system does not provide any functionality to clean the garbage that 
the data servers keep stored.

After creating a file, the client can open it. In this case, the metadata 
server sends the client all the information it has stored for that file.
The metadata server count how many clients are using each file, increases
this number when a client open and decreases when it close.
That way, the client is able to perform read and write operations. While 
the file is open, it cannot be copied if the data server that holds the 
file is not available.

The primary metadata, from time to time, pings the data servers to know 
which are available and which are not. When a data server is not 
available, the metadata server will try to migrate the files stored on 
that server to another one. The new server should be on the available list of
servers to avoid choosing the same. The copy of the files should happen only
if the file being copied isn't open. When the file is fully copied to other 
server, meaning the metadata already sent the request to it to create a 
new file and it did respond saying the file was created, the metadata 
server has to update the list of servers that hold that file. The downside 
of this process is that the file is no longer referenced to the 
old server but the file is still there and there's is not a proper way 
to remove it.

\subsubsection{Failure and Recovery of Metadata Servers}

All the replicas need to know if the primary replica is well and good to 
keep executing the requests that clients are making. To keep track of this, 
the primary server sends a heartbeat along with a log of the servers that 
are available and unavailable so they can update their own list. Supposing 
the primary replica is down and can't send the other replicas the heartbeat 
they're expecting, the replicas need to cover the primary so the client can 
keep asking for create, open or close operations. To do this, the replicas 
send each other their ID's and the one with the lower ID is the new primary
replica. If a replica does not receive any response from the other, it 
means that it is not available either and this replica has no choice but to 
be the new primary server. When a replica recovers, asks the others who is 
the primary server and updates its state accordingly. During all this process, 
metadata servers can't respond to any request from the clients.

%------------------------------------------------------------------------- 
\subsection{Data Server}

Data Servers store the content of one or more files. The content of each
file is stored using the local file system, in a local file with a
16-character ASCII string name generated by the Metadata Server. The
state and all files of the Data Server are written to disk.

Data Servers store files with a version composed by a
monotonically increasing number, that starts with 0, and a timestamp.
The files also have a byte array with the content of the file.

\subsubsection{Data Servers and Metadata Server}

The first action of a Data Server is to inform the Metadata Servers that
it is available to receive requests to store files from. The Metadata
Server should register the Data Server as available and reply to the Data
Server that the registration took success. The Data Server waits for the
reply and if it does not receive any, it resend the message. The secundary
Metadata Servers discard this message.

The Data Server sends a regular message, heart beat, to all the Metadata
Servers. The message is ignored by the two replicas, whether they are
available or not. This way, the Data Server does not have to keep
registered which Metadata Server is the primary and simplifies when
there was a change of primary Metadata Server, the Data Server does not
have to resend the message to the new primary.
This heart beat is useful to the primary Metadata Server keep track of
which servers are available to store new files and to avoid clients
from beeing waiting for Data Servers to come up to have a majority quorum.
We choose to not give the responsability to the Metadata Server of ask to
any Data Server if it is alive and wait for an answer because, in a large
system, the workload of the Data Servers will be lower compared to the
workload of the Metadata Servers. The exception is when multiple clients
are reading and writing intensively to the same few Data Servers. We
understood that this is an exception case. With this aproach, only one
message is sent from the Data Server to the Metadata Server trough the
network.

When the Data Server fails or freezes it stops sending heart beats to
the Metadata Servers. If the primary Metadata Server does not receive an
heart beat in a time window, it believes the Data Server is unavailable
and this servers stops receiving requests from de Metadata Server to
create new files.

Each time the Metadata Server asks the Data Server to create a new file
with a given name, the Data Server tries to create the new file and answers
to the Metadata Servers if the operation took success or not.

The Metadata Server may request one Data Server to send a copy of one file
to another Data Server. This request is treated as any other request.

\subsubsection{Reading and Writing to Data Servers}

Clients ask Data Servers to read or write files. This requests are
processed as they arrive to the server without having any priority neither
any reordering. It is simpler to buffer the requests as they arrive and to
perform them later.

Every file written in Data Servers have a version composed by a version
number that is incremented everytime the file is written and a timestamp
from the last writing. This timestamp let the Data Server to know if it is
receiving a version earlier than the version it stores.

When the Data Server is requested to store a new version of the file, it
checks if the new version is newer than the version that it stores, it
compares the timestamps and if it is earlier, it discards the version, else,
increments the version number of the file and stores it to disk.
This solution raises the problem of destroying the previous work of one
client when other, that read a previous file and edited it, asks the server
to write its version, overwriting the previous version, written by other
client, without updating its version with the content from the previous. To
avoid this, we could implement a solution similar to distributed revision
control systems, which would rise the complexity of the project.

%------------------------------------------------------------------------- 
\subsection{Client}

The client can create, open, read, write, close and delete files. When  it is created, it is provided the contact of the all metadata servers and which of these is primary server. 
The client can contact the primary metadata server to create, open, close or delete a file.

\subsubsection{Create} 

When client request to create a new file to the metadata servers, it specifies the name of the file, the number of data servers used to store the data and the number of data server that it needs to obtain a read and write quorum. The size of quorum can't exceed the number of data server used to store the file. 

\subsubsection{Open}

If client wants to open a file, it sends a request to primary metadata server to open a file, and obtains the information about the data servers where is stored the file content, the local name of these files in data servers and the size of read and write quorum.

\subsubsection{Read and Write}

In Reads and Writes, the client contacts directly data server, since it client already knows where is the file. During the write of the file in data servers, the client must block until receives the confirmation from the quorum of data servers. The size of this quorum is defined in metadada servers, when the file is created. Like in write, in read the client also block until the data server sends a response with read quorum.
If while the process request to read or write a file, the number of data servers that answer is less than required to obtain quorum, due to communication failures or if server can't answer, the client is put on hold and continues to try contact the data servers until get that quorum.
In the event of data server freeze, the messages that client sends are buffered. When it unfreezes, the server sends an answer for each request that is in the buffer. In this case, the client just consider the first answer, and it ignoring the other ones.
The client doesn't control the file version number. This number is controlled by data server.

\begin{itemize}
\item Default Read - The client makes a request to data server to read a file,  wait for majority quorum, and accept  that file even it is an older version.

\item Monotonic Read - The client makes a request to data server to read a file and waits for majority quorum to retrieve  a version equal or greater than.
\end{itemize}



%------------------------------------------------------------------------ 
\subsection{Formatting your paper}

All text must be in a two-column format. The total allowable width of 
the text area is 6-7/8 inches (17.5 cm) wide by 8-7/8 inches (22.54 cm) 
high. Columns are to be 3-1/4 inches (8.25 cm) wide, with a 5/16 inch 
(0.8 cm) space between them. The main title (on the first page) should 
begin 1.0 inch (2.54 cm) from the top edge of the page. The second and 
following pages should begin 1.0 inch (2.54 cm) from the top edge. On 
all pages, the bottom margin should be 1-1/8 inches (2.86 cm) from the 
bottom edge of the page for $8.5 \times 11$-inch paper; for A4 paper, 
approximately 1-5/8 inches (4.13 cm) from the bottom edge of the page.

%------------------------------------------------------------------------- 
\subsection{Type-style and fonts}

Wherever Times is specified, Times Roman may also be used. If neither is 
available on your word processor, please use the font closest in 
appearance to Times that you have access to.

MAIN TITLE. Center the title 1-3/8 inches (3.49 cm) from the top edge of 
the first page. The title should be in Times 14-point, boldface type. 
Capitalize the first letter of nouns, pronouns, verbs, adjectives, and 
adverbs; do not capitalize articles, coordinate conjunctions, or 
prepositions (unless the title begins with such a word). Leave two blank 
lines after the title.

AUTHOR NAME(s) and AFFILIATION(s) are to be centered beneath the title 
and printed in Times 12-point, non-boldface type. This information is to 
be followed by two blank lines.

The ABSTRACT and MAIN TEXT are to be in a two-column format. 

MAIN TEXT. Type main text in 10-point Times, single-spaced. Do NOT use 
double-spacing. All paragraphs should be indented 1 pica (approx. 1/6 
inch or 0.422 cm). Make sure your text is fully justified---that is, 
flush left and flush right. Please do not place any additional blank 
lines between paragraphs. Figure and table captions should be 10-point 
Helvetica boldface type as in
\begin{figure}[h]
   \caption{Example of caption.}
\end{figure}

\noindent Long captions should be set as in 
\begin{figure}[h] 
   \caption{Example of long caption requiring more than one line. It is 
     not typed centered but aligned on both sides and indented with an 
     additional margin on both sides of 1~pica.}
\end{figure}

\noindent Callouts should be 9-point Helvetica, non-boldface type. 
Initially capitalize only the first word of section titles and first-, 
second-, and third-order headings.

FIRST-ORDER HEADINGS. (For example, {\large \bf 1. Introduction}) 
should be Times 12-point boldface, initially capitalized, flush left, 
with one blank line before, and one blank line after.

SECOND-ORDER HEADINGS. (For example, {\elvbf 1.1. Database elements}) 
should be Times 11-point boldface, initially capitalized, flush left, 
with one blank line before, and one after. If you require a third-order 
heading (we discourage it), use 10-point Times, boldface, initially 
capitalized, flush left, preceded by one blank line, followed by a period 
and your text on the same line.

%------------------------------------------------------------------------- 
\subsection{Footnotes}

Please use footnotes sparingly%
\footnote
   {%
     Or, better still, try to avoid footnotes altogether.  To help your 
     readers, avoid using footnotes altogether and include necessary 
     peripheral observations in the text (within parentheses, if you 
     prefer, as in this sentence).
   }
and place them at the bottom of the column on the page on which they are 
referenced. Use Times 8-point type, single-spaced.


%------------------------------------------------------------------------- 
\subsection{References}

List and number all bibliographical references in 9-point Times, 
single-spaced, at the end of your paper. When referenced in the text, 
enclose the citation number in square brackets, for example~\cite{ex1}. 
Where appropriate, include the name(s) of editors of referenced books.

%------------------------------------------------------------------------- 
\subsection{Illustrations, graphs, and photographs}

All graphics should be centered. Your artwork must be in place in the 
article (preferably printed as part of the text rather than pasted up). 
If you are using photographs and are able to have halftones made at a 
print shop, use a 100- or 110-line screen. If you must use plain photos, 
they must be pasted onto your manuscript. Use rubber cement to affix the 
images in place. Black and white, clear, glossy-finish photos are 
preferable to color. Supply the best quality photographs and 
illustrations possible. Penciled lines and very fine lines do not 
reproduce well. Remember, the quality of the book cannot be better than 
the originals provided. Do NOT use tape on your pages!

%------------------------------------------------------------------------- 
\subsection{Color}

The use of color on interior pages (that is, pages other
than the cover) is prohibitively expensive. We publish interior pages in 
color only when it is specifically requested and budgeted for by the 
conference organizers. DO NOT SUBMIT COLOR IMAGES IN YOUR 
PAPERS UNLESS SPECIFICALLY INSTRUCTED TO DO SO.

%------------------------------------------------------------------------- 
\subsection{Symbols}

If your word processor or typewriter cannot produce Greek letters, 
mathematical symbols, or other graphical elements, please use 
pressure-sensitive (self-adhesive) rub-on symbols or letters (available 
in most stationery stores, art stores, or graphics shops).

%------------------------------------------------------------------------ 
\subsection{Copyright forms}

You must include your signed IEEE copyright release form when you submit 
your finished paper. We MUST have this form before your paper can be 
published in the proceedings.

%------------------------------------------------------------------------- 
\subsection{Conclusions}

Please direct any questions to the production editor in charge of these 
proceedings at the IEEE Computer Society Press: Phone (714) 821-8380, or 
Fax (714) 761-1784.

%------------------------------------------------------------------------- 

\nocite{ex1,ex2}
\bibliographystyle{latex8}
\bibliography{latex8}

\end{document}

