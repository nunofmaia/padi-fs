\subsection{Metadata Server}

Metadata servers maintain metadata information about the files stored
in the data servers. This metadata holds relevant information about
the files such as the file name, the number of servers used to store
the contents of the file, the minimum number of data server that should
be contacted during the read and write operations and a list of the
data servers that have the file stored. The metadata server also stores
information about the number of files each data server stores, which
servers are available or not and which files are open in clients.
The reason why we do this will be explained below on document.

As it was previously mentioned, there are 3 replicas of the metadata
server. The purpose of these replicas is to have the content of the
server replicated so that when a replica fails, there's another one
that can reply to the requests made. This replication is made in a
passive manner, which means that just one of the replicas will be
responsible for all the operations. The primary replica has to
synchronize its state with the secondary ones to make sure that all
the replicas store the same information at all time. Passive replication
is more useful in this system than active replication in a way that we
can tolerate two failures and have always a server to be able to reply to
any request made by the client.

\subsubsection{Handling Files}

The primary server can reply to several requests from the client. This
request are create, open, close and delete operations.
When a client send a request to create a file, the server saves a new
registry for that file with the information sent with the request. The
primary replica must inform the necessary data servers that they need to
create a local copy of the file and wait for them to reply back. Then
it increases the number of files for each server and sends the new 
entry to the other two replicas so they can be up to date. It's important 
to mention that when a data server is down, the metadata server will try 
to send the request to another server. If there are no more servers to 
send the request to, the metadata server will wait for the data server 
to come up to send it the information it needs to create the file. A 
secondary replica can also be down when the primary one sends its new 
information. When this happens, the primary server will not wait for 
the reply of the other server. Just after this exchange of requests and
replies, the metadata server tells the client it's all good and the file 
was successfully created.

When a client asks to delete a file, the metadata server decreases the
number of files for each server that holds the file and removes the 
corresponding entry from the table where all the information is stored. 
The system does not provide any functionality to clean the garbage that 
the data servers keep stored.

After creating a file, the client can open it. In this case, the metadata 
server sends the client all the information it has stored for that file.
The metadata server count how many clients are using each file, increases
this number when a client open and decreases when it close.
That way, the client is able to perform read and write operations. While 
the file is open, it cannot be copied if the data server that holds the 
file is not available.

The primary metadata, from time to time, pings the data servers to know 
which are available and which are not. When a data server is not 
available, the metadata server will try to migrate the files stored on 
that server to another one. The new server should be on the available list of
servers to avoid choosing the same. The copy of the files should happen only
if the file being copied isn't open. When the file is fully copied to other 
server, meaning the metadata already sent the request to it to create a 
new file and it did respond saying the file was created, the metadata 
server has to update the list of servers that hold that file. The downside 
of this process is that the file is no longer referenced to the 
old server but the file is still there and there's is not a proper way 
to remove it.

\subsubsection{Failure and Recovery of Metadata Servers}

All the replicas need to know if the primary replica is well and good to 
keep executing the requests that clients are making. To keep track of this, 
the primary server sends a heartbeat along with a log of the servers that 
are available and unavailable so they can update their own list. Supposing 
the primary replica is down and can't send the other replicas the heartbeat 
they're expecting, the replicas need to cover the primary so the client can 
keep asking for create, open or close operations. To do this, the replicas 
send each other their ID's and the one with the lower ID is the new primary
replica. If a replica does not receive any response from the other, it 
means that it is not available either and this replica has no choice but to 
be the new primary server. When a replica recovers, asks the others who is 
the primary server and updates its state accordingly. During all this process, 
metadata servers can't respond to any request from the clients.