\subsection{Client}

The client can create, open, read, write, close and delete files. When  it is created, it is provided the contact of the all metadata servers and which of these is primary server. 
The client can contact the primary metadata server to create, open, close or delete a file.

\subsubsection{Create} 

When client request to create a new file to the metadata servers, it specifies the name of the file, the number of data servers used to store the data and the number of data server that it needs to obtain a read and write quorum. The size of quorum can't exceed the number of data server used to store the file. 

\subsubsection{Open}

If client wants to open a file, it sends a request to primary metadata server to open a file, and obtains the information about the data servers where is stored the file content, the local name of these files in data servers and the size of read and write quorum.

\subsubsection{Read and Write}

In Reads and Writes, the client contacts directly data server, since it client already knows where is the file. During the write of the file in data servers, the client must block until receives the confirmation from the quorum of data servers. The size of this quorum is defined in metadada servers, when the file is created. Like in write, in read the client also block until the data server sends a response with read quorum.
If while the process request to read or write a file, the number of data servers that answer is less than required to obtain quorum, due to communication failures or if server can't answer, the client is put on hold and continues to try contact the data servers until get that quorum.
In the event of data server freeze, the messages that client sends are buffered. When it unfreezes, the server sends an answer for each request that is in the buffer. In this case, the client just consider the first answer, and it ignoring the other ones.
The client doesn't control the file version number. This number is controlled by data server.

\begin{itemize}
\item Default Read - The client makes a request to data server to read a file,  wait for majority quorum, and accept  that file even it is an older version.

\item Monotonic Read - The client makes a request to data server to read a file and waits for majority quorum to retrieve  a version equal or greater than.
\end{itemize}
