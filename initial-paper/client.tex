\subsection{Client}

The client can create, open, read, write, close and delete files. When
it is created, it is provided the contact of the all metadata servers
and one of these is a primary server. 
The client can contact the primary metadata server to create, open,
close or delete a file.

\subsubsection{Create} 

When a client requests the metadata server to create a new file, it
specifies the name of the file, the number of data servers used to
store the file and the number of data servers it needs to obtain a read
and write quorum. The size of quorums can not exceed the number of data
servers used to store the file. 

\subsubsection{Open}

If a client wants to open a file, it sends a request to the primary
metadata server to open it, the primary answers with the information
about the data servers where is stored the file content, the local name
of these file and the size of read and write quorums.

\subsubsection{Read and Write}

When reading or writing, the client contacts directly the data server,
since its client already knows where is the file. During the write of
the file in data servers, the client must block until it receives the
confirmation from the quorum. The size of this quorum is defined in
metadada servers, when the file is created. Like in write, in read the
client also block until the data server sends a response with read quorum. ??
If the client requests to read or write a file and the number of data
servers that answer is less than required to obtain quorum, due to
communication failures or if server can't answer, the client is put on
hold and continues to try contact the data servers until it gets a
majority quorum.
In the event of data server freeze, the messages that client sends are
buffered. When it unfreezes, the server sends an answer for each request
that is in its buffer. In this case, the client just consider the first
answer, and it ignores the other ones.
The client doesn't control the file version number. This number is
controlled by data server.

\begin{itemize}
\item Default Read - The client makes a request to data server to read
a file, wait for majority quorum, and accept that file even it is an
older version.

\item Monotonic Read - The client makes a request to data server to read
a file and waits for majority quorum to retrieve a version equal or
greater than.
\end{itemize}
