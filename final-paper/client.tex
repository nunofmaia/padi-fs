\subsection{Client}

The client can create, open, read, write, close and delete files. When
it is created, it is provided the contact of the all metadata servers
and one of these is a primary server. 
The client can contact the primary metadata server to create, open,
close or delete a file.

\subsubsection{Create} 

When a client requests the metadata server to create a new file, it
specifies the name of the file, the number of data servers used to
store the file and the number of data servers it needs to obtain a read
and write quorum. The size of quorums cannot exceed the number of data
servers used to store the file.

\subsubsection{Open}

If a client wants to open a file, it sends a request to the primary
metadata server to open it and the primary server answers with a list
of data servers where the file is stored, the local name of the file and 
the size of read and write quorums.

\subsubsection{Read and Write}

When reading and writing, the client contacts directly with the data server,
since it already knows where is the file. During the write of
the file in the data servers, the client must block until it receives the
confirmation from the quorum. The size of this quorum is defined in
metadata servers when the file is created. Like in write operation, when the
client performs a read operation it also blocks until the data servers send 
back a response and then achieve a majority quorum.
If the client requests to read or write a file and the number of data
servers that answer is less than required to obtain quorum, due to
communication or server failures, the client is put on
hold and continues to try to contact the data servers until it gets a
majority quorum.
In the event of a data server freeze, the messages that a client sends are
buffered. When the server unfreezes, it sends an answer for each request
that is in its buffer. In this case, the client considers just the first
answer, ignoring the other ones.
The client doesn't control the file version number. This number is
controlled by data server.

\begin{itemize}
\item Default Read - The client makes a request to the data server to read
a file, waits for a majority quorum and accepts the file even it is an
older version.

\item Monotonic Read - The client makes a request to the data server to read
a file and waits for a majority quorum to retrieve a file with a version equal or
greater than the version of the last file read.
\end{itemize}
