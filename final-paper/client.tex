\subsection{Client}

The client can create, open, read, write, copy, close and delete files. When
it is created, it is provided the contact of the all metadata servers
and one of these is a primary server. 
The client can contact the primary metadata server to create, open,
close or delete a file.

\subsubsection{Create} 

When a client requests the metadata server to create a new file, it
specifies the name of the file, the number of data servers used to
store the file and the number of data servers it needs to obtain a read
and write quorum. The size of quorums cannot exceed the number of data
servers used to store the file. In response, the metadata server sends back the
metadata of the file that the clients save in a list of open files.

\subsubsection{Open}

If a client wants to open a file, it sends a request to the primary
metadata server to open it and the primary server answers with the file
metadata which has the list of data servers where the file is stored, the local name
of the file and the size of read and write quorums. Like in the create operation, the
metadata server sends back the metadata of the requested file.

\subsubsection{Read and Write}

For reading and writing files, the client needs to contact with the data servers
that hold that file in their local disk. During this operations, the client blocks
itself until it receives a confirmation from the data servers. The number of data
servers needed for the client to know that it's OK to go on with its request is known
as a quorum. The quorum for reads and writes is defined on the creation of each file.
If the quorum is not reached, because the number of data servers that have the file
is not enough or due to communication problems or server failures, the client tries
to make the request again after a certain time, fetching once again the file's metadata
in case the metadata had been updated in the meantime.

When performing writes, to updated the version number of a file, the client needs to
request a new sequence number to the metadata so that the version can be greater
than the old one.

Reading operations can have a different behaviour depending on its type. There are
two types of reading operations: 

\begin{itemize}
\item Default Read - The client makes a request to the data server to read
a file, waits for a majority quorum and accepts the file even it is an
older version.

\item Monotonic Read - The client makes a request to the data server to read
a file and waits for a majority quorum to retrieve a file with a version equal or
greater than the version of the last file read.
\end{itemize}

It's also worth mentioning that clients have two buffers, a string buffer and a file
register and each one of this buffers have a file associated. The string buffer is
responsible to save the file's contents while the file register holds the metadata
of a given file.

Clients can also perform copy operations where it can take the contents of a
given file and copy them to another one.



