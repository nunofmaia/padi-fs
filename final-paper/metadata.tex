\subsection{Metadata Server}

Metadata servers maintain metadata information about the files stored
in the data servers. This metadata holds relevant information about
the files such as the file name, the number of servers used to store
the contents of the file, the minimum number of data servers that should
be contacted during the read and write operations and a list of the
data servers that have the file stored. This servers also keep track of
other replicas that may exist, which data servers are available or not as
well as the number of files each one of them holds, the clients that can
interact with the system and which files they have open and files that have
to be updated because there are not enough data servers to answer to its
requests. The reason why we keep knowledge of this things and more
will be explained below on document.

\subsubsection{Replication}

As it was previously mentioned, there are 3 replicas of the metadata
server. The purpose of these replicas is to have the content of the
server replicated so that when a replica fails, there's another one
that can reply to the requests made. This replication is made in a
passive manner, which means that just one of the replicas will be
responsible for all the operations. Passive replication
is more useful in this system than active replication in a way that we
can tolerate two failures and have always a server to be able to reply to
any request made by the client. At any moment in time, the primary replica
can fail and another one has to replace it but for this to happen without
any problems, the replicas need to be synchronized. The primary replica is
responsible for sending some kind of instruction to the other replicas for
them to have the same state as the primary. To accomplish this, each replica
logs the command issued by the primary one and behaves accordingly. As we
go further in this paper, we'll explain how the log plays an important goal
when a server recovers.

\subsubsection{Handling Files}

The primary server can reply to several requests from the client. This
requests are create, open, close and delete operations.

\medskip
\textbf{Create}
\smallskip 

When a client sends a request to create a file, the server saves a new
registry for that file with the information sent with the request. The
server informs the available data servers needed with the least amount
of saved files to save a local copy of the file, updates the load
of each data server that will hold that file and sends the file's metadata
to the client. When the client requests for the file to be replicated in
a certain number of data servers and that number is greater than the 
data servers available at that moment, the metadata server sends the 
file's metadata to the client with the data servers available and puts the 
file on a pending lists so that when a new data server registers, the metadata 
of the file can be updated and sent back to the client. The file stays in the 
pending list until all the data servers requested are available.

After a new data server registers itself with the metadata server, the latter
is responsible for updating the metadata of files currently on the pending
list, adding the data server to their list and removing the files if they already
have the number of data servers needed, and for sending the create instruction
to the data server.

%The
%primary replica must inform the necessary data servers that they need to
%create a local copy of the file and wait for them to reply back. Then
%it increases the number of files for each server and sends the new 
%entry to the other two replicas so they can be up to date. It's important 
%to mention that when a data server is down, the metadata server will try 
%to send the request to another server. If there are no more servers to 
%send the request to, the metadata server will wait for the data server 
%to come up to send it the information it needs to create the file. A 
%secondary replica can also be down when the primary one sends its new 
%information. When this happens, the primary server will not wait for 
%the reply of the other server. Just after this exchange of requests and
%replies, the metadata server tells the client it's all good and the file 
%was successfully created.

\bigskip
\textbf{Open}
\smallskip

After creating a file, the client can operate on it right away. But when
the file is closed, the client can request to open it. In this case, the
metadata server sends to the client the metadata stored for that file. The
server keeps record of what clients opened a certain file, so when a client
opens a file, the metadata server adds it to the list of the corresponding
file.

%After creating a file, the client can open it. In this case, the metadata 
%server sends the client all the information it has stored for that file.
%The metadata server has a counter per file to know how many users have 
%that file opened. While the file is open, it cannot be copied if the data 
%server that holds the file is not available.

\bigskip
\textbf{Close}
\smallskip

A client with an open file can close it whenever it decides to. The close
operation requires the metadata server to check open files to be able to
remove that client from the list of clients with that file open.

\bigskip
\textbf{Delete}
\smallskip

When a client asks to delete a file, the metadata server decreases the
number of files for each server that holds the file and removes the 
corresponding entry from the table where all the metadata for that file is stored.
If the file is open when the client tries to delete it, the server needs
to remove the file entry from the open files as well.
Although the file's entries are removed from the metadata server, they
still remain untouched in the data server since the system does not provide
any functionality to clean deleted files from them.

\bigskip

There are other file operations that don't have direct participation from the
metadata servers like read, write and copy operations. However, metadata servers
play an important role when this operations occur. The primary replica has a sequencer
that is responsible for deliver a unique version number to the client when it performs
write operations to the data servers. This mechanism ensures that any new version of a
file will have a greater number than older versions. This sequence number will be
synchronized to the other replicas so that, when the old primary fails the new one can
still continue the sequence of numbers started before.

\subsubsection{Migration}

From time to time, the primary replica pings the data servers to know
which are available and which are not. The data server's response to the ping
brings statistical information about that server - the number of accesses to
the server. The number of accesses to the server correspond to the number
of accesses made to each file that the server holds. When this number is
starting to be to high, we need to consider the migration of the files to another
data server to have a more evenly distribution of accesses across all the data
servers available. We propose a migration algorithm that we believe it is able
to perform its task in a very simple manner and at the same time maintain
a good balance between the accesses to each data server. The first part of the
algorithm can be described by the following steps:

\begin{enumerate}

\item Calculate the sum of the accesses to each file of each data server;
\item Calculate the average of the values calculated in the step before;
\item Using the average of accesses to each data server, define an interval with
a margin of 20\%;
\item Finally, list all the data servers that are currently off the interval

\end{enumerate}

After this steps, we need to know which server, from the list of servers that do not
fit in the specified interval, is the most overloaded and which server, from the list
of servers that do not fit the interval, is the one with the least number of files.

Having those servers selected, we begin to choose closed files. Most people would
think that the most accessed file should be the one to be moved to another server
but the probability of that file to keep being accessed is too high which means that
we would have to move this file from server to server for a long time. Having that said,
we should begin by picking closed files starting by the second most accessed file from
the most accessed server on the list. Those files are transfered to the server with
the least number of accesses and then verified to check if everything went fine. It is
important to remember once again that for the files to be transfered to another
server they can't be open by any client meaning that only files that are not being used
at that moment can be moved.

After the migration, the metadata server updates the metadata of the files transfered
so that when the clients open them again, they have an updated list of where the files
are located.

%The primary metadata, from time to time, pings the data servers to know 
%which are available and which are not. When a data server is not 
%available, the metadata server will try to migrate the files stored on 
%that server to another one. The new server should be on the available 
%can't already have stored the file that needs to be copied. The 
%copy of the files should happen only if the file being copied isn't open. When 
%the file is fully copied to other server, meaning the metadata already sent 
%the request to an available server that has the file stored to copy it to the new 
%server and did receive the response saying the file was copied, the metadata 
%server has to update the list of servers that hold that file. The downside 
%of this process is that the file is no longer referenced to the old server 
%but the file is still there and there's is not a proper way to remove it.

\subsubsection{Failure and Recovery}

The replicas need to know if the primary replica is well and good to keep
executing the requests being made or else the only time the replicas would
receive some knowledge of the primary being alive is when they receive a
command to update their log and execute. To bypass this situation, each
replica asks the primary replica if it is alive.

In order to keep everything as synchronized as possible, metadata servers use two
ways of knowing their state at a certain point in time - a log and periodic backups.

\bigskip
\textbf{Log}
\smallskip

We already mentioned that, for synchronization purposes, each replica keeps a log
where it appends commands issued by the primary replica that represent important
changes to its state. One possible approach to synchronize all the replicas was to
send all the data stored by the primary replica to the other replicas but this
information can become very large and pass this chunck of data through the network
was not the best way to do it. So, we took the other direction and chose the log
implementation. This approach is must lighter than the other in the sense that only
a string command is sent over the network and, when the command is received, each
replica runs that same command on its own, updating their individual state.

\bigskip
\textbf{Backup}
\smallskip

The log helps the metadata server holding an history of what happened during the time
it is running but we still find that having a backup in disk of the server's state is really
helpful too. So, on a specified interval, the metadata server writes to disk a backup with
all the data that it may need if something wrong happens.

\bigskip

Supposing the primary replica fails, it can't send an heartbeat to the other replicas. When
those same replicas don't receive an answer from the primary one, they know that something
wrong happend and they need to decide which replica will cover the main replica and
begin accepting requests. In our implementation, the replica with the lowest ID will be
the next primary replica. Since the log from the replica has the same length and content as
the former primary server, the system continues to run normally.

\medskip

Eventually, replicas will recover and start running again. When this happens, the replica
will be outdated, meaning that there is a high probability that the primary replica had
already made a few more operations while the other replica was down. The recovered
replica needs to update itself to the same state has the others. To prevent the primary
replica to send the full log for the replica to execute, the replica gets the most recent
backup from disk first to recreate its previous state and only after asks the primary
server to send to it the missing lines from the log. With this method, we can lower the
time spent on the recovering process since the replica only needs to run a very few
number of commands comparing this number with the total of commands in the log.
